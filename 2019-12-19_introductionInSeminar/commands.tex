%!TEX root = ../2019-12-17_introductionInSeminar.tex
% Animated bubble diagram
\tikzset{%
    Partha/.cd,
    start angle/.initial=0,
    orientation/.initial=1
}
\makeatletter
\newlength{\bubbleInitAngle}
\newcommand{\BubbleDiagramAnimated}[2][]{%
    \begin{tikzpicture}[%
        every node/.style={align=center,let hypenation},
        Partha/.cd,#1
    ]
    \setlength{\bubbleInitAngle}{%
        \ifnumequal{\pgfkeysvalueof{/tikz/Partha/orientation}}{1}{0pt}{-180pt}
    }
    \foreach \smitem [count=\xi] in {#2}{\global\let\maxsmitem\xi}
    \pgfmathtruncatemacro\actualnumitem{\maxsmitem-1}
    \foreach \smitem [count=\xi] in {#2}{%
        \ifnumequal{\xi}{1}{ %true
            \node[bubble center node, smvisible on=<\xi->](center bubble){\smitem};
        }{%false
            \pgfmathtruncatemacro{\xj}{\xi-1}
            \pgfmathtruncatemacro{\angle}{
                \pgfkeysvalueof{/tikz/Partha/start angle}
                + (\pgfkeysvalueof{/tikz/Partha/orientation} * 360 / \actualnumitem * \xj)
                + \bubbleInitAngle
            }
            \edef\col{\@nameuse{color@\xj}}
            \node[bubble node, smvisible on=<\xi->](module\xi) at (center bubble.\angle) {\smitem};
        }%
    }%
    \end{tikzpicture}
}
\makeatother

% Fullscreen frames
\newcommand{\plainFrame}[1]{%
    {
        \setbeamertemplate{headline}{\vskip-\headheight}
        \setbeamertemplate{footline}{\defaultfootline}
        \begin{frame}
            \vspace{-\headheight} % FIXME headline seems not to work as expected
            #1
        \end{frame}
    }
}
\newcommand{\imageFrame}[2][0pt]{%
    {
        \setbeamertemplate{headline}{}
        \setbeamertemplate{footline}{\defaultfootline}
        \begin{frame}
            \begin{textblock*}{\pagewidth} (0pt, #1)
                \centering
                \includegraphics[width=\pagewidth,height=\pageheight,keepaspectratio]{#2}
            \end{textblock*}
        \end{frame}
    }
}
\newcommand{\imageFrameSvg}[1]{%
    {
        \setbeamertemplate{headline}{}
        \setbeamertemplate{footline}{\defaultfootline}
        \begin{frame}
            \begin{textblock*}{\pagewidth} (0pt, 0pt)
                \centering
                \includesvg[width=\textwidth,height=\textheight]{#1}
            \end{textblock*}
        \end{frame}
    }
}
\newcommand{\videoFrame}[2]{%
    {
        \setbeamertemplate{headline}{\vskip-\headheight}
        \setbeamertemplate{footline}{\defaultfootline}
        \begin{frame}
            \centering
            \Huge{\href{run:#1}{#2}}
        \end{frame}
    }
}
\newcommand{\draftFrame}[1]{%
    \begin{frame}[plain]
        \centering
        {\Huge \draft{#1}}
    \end{frame}
}

% Subsection color box
\makeatletter
\newcommand*{\currentname}{\@currentlabelname}
\makeatother

\newcommand<>{\subsectionBox}[1]{%
    \uncover#2{%
        \begin{tikzpicture}[remember picture,overlay]%
            \draw[fill=#1, draw=none]
                (current page.north west) rectangle ++(20pt, -\pageheight);
            \node[anchor=south west] at (current page.south west)
                {\large \rotatebox{90}{\currentname}};
        \end{tikzpicture}
    }
}

% Short codes
\newcommand{\orcid}[1]{\href{https://orcid.org/#1}{\textcolor[HTML]{A6CE39}{\aiOrcid}}}
\newcommand{\drivebuild}{{\scshape DriveBuild}}
\newcommand{\Eg}{E.\,g.\xspace}
\newcommand{\draft}[1]{\color{red} \textit{#1}}
\newcommand{\draftItemize}[1]{%
    \begin{itemize}
        \color{red}
        \itshape{}
        #1
    \end{itemize}
}
\newcommand{\sectionFrame}{%
    \begin{frame}
        \vfill
        \centering
        \begin{beamercolorbox}[sep=8pt,center,shadow=true,rounded=true]{title}
            \usebeamerfont{title}\insertsectionhead\par%
        \end{beamercolorbox}
        \vfill
    \end{frame}
}

