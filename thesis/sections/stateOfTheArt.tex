%!TEX root = ../thesis.tex
\section{State of the Art}\label{sec:stateOfTheArt}
\Cref{sec:problemStatement} shows that simulation based testing of \glspl{av} is a complex task and there are many problems which have to be investigated.
The following subsections explain and discuss current approaches for solving some of these problems.

\subsection{Formalization of Environments and Criteria}
Concerning the definition of test environments \opendrive{}~\cite{openDrive} is one of most popular formats for defining very comprehensive and very detailed environments.
Many well known car manufacturers (\eg{} Audi, \gls{bmw} and Daimler) and other organizations like Fraunhofer, \gls{tum} and \gls{dlr} Institute of Robotics and Mechatronics use it.
The format is \gls{xml} based and offers declarations for many different kinds of objects like signs, cross falls, rail roads, bridges and signals which may even dynamically change.
Especially the definition of streets and rail roads can be very complex.
Additionally these object can be enhanced with meta information.
So \opendrive{} allows to define predecessor and successor lanes, neighbor lanes, complex junctions, parking spaces, acceleration strips, side walks, multiple different types of markings, reference lines for roads/junctions and rail road switches.
Although \opendrive{} has many options and capabilities to define environment elements \opendrive{} on its own has neither the capability to add traffic participants nor to specify their movements nor to express any kind of test criterion.\\
\opencrg{}~\cite{openCrg} is a \gls{xml} based format which extends \opendrive{}.
It allows to increase the details of the roads defined in \opendrive{} by adding bumps and unevenness to road surfaces using \glspl{crg}.
Therefore \opencrg{} already includes tons of predefined data measured on existing roads.\\
\openscenario{}~\cite{openScenario} is a \gls{xml} based scheme to add traffic participants to \opendrive{} scenarios and bundles them with their physical properties and their dynamic behavior.
The behavior is organized in maneuvers which are sequences of abstract actions like change lane, brake, accelerate and adapt the distance to other participants.
\openscenario{} is capable to define conditions which trigger these maneuvers as soon as they are satisfied.
The variety of conditions includes \gls{ttc}, time headaway, (relative) speed, traveled distance, speed, acceleration or reaching a certain position.
Since \openscenario{} uses \gls{xml} for the description of the dynamic behavior it can not change during the simulation.
Further maneuvers can not do any computations throughout a simulation \eg{} calculate steering angles or any other information which the underlying simulator does not directly provide.
\openscenario{} is not able to define any kind of test criteria as well.\\
Another very popular format for declaring environments is \commonroad{}~\cite{commonRoad} which focuses solely on path planning problems.
\commonroad{} scenarios are only capable to define lanes, obstacles and cars.
A car can be associated with a list of states that describe its movement.
Each state consists of a logical timestamp and the current position, orientation and speed of the car.
The speed as well as the position may be not specified exactly but with an interval which allows to formulate uncertainty of these attributes.
However, since \commonroad{} bundles timestamps with positions it does not guaranty that specified movements are realistic.
Hence it is needed to check feasibility of movements separately beforehand.
The definition of lanes in \commonroad{} is based on lanelets~\cite{lanelets} which consist of two sequences of points that describe the left and the right border of a lane.
In contrast the simulator I will use in my work (See \cref{sec:testCycle}) describes lanes with a sequence of road center points, the current width at each point and the number of left and right lanes.
The simulator interpolates the sequence of road center points as well as the widths to generate a smooth curvature.
As a result the simulator can visualize arbitrary lanelets of \commonroad{} scenarios with a guaranteed high accuracy.
Concerning the definition of test criteria \commonroad{} is restricted to the definition of simple goal regions.
If an \gls{av} reaches a goal region the test is considered successful and it failed otherwise.\\
There is also a model based approach to describe scenarios~\cite{modelBasedDescription}.
In contrast to other approaches this approach focuses on creating a description scheme that is not only comprehensive but also human-readable and abstracts from a scenario to a logical level.
A main point is the abstraction in terms of the separation of spatial and temporal information.
This separation allows the definition of acts which describe the current situation at some point during a test.
Every act defines exactly one event which triggers a transition to another act in order to create a linear sequence of acts.
All cars in a test have multiple perception layers of different sizes which surround the car.
These perception layers define event triggers.
The movements of participants are specified based on a predefined set of maneuvers.
So the description of behaviors of \glspl{av} and the relation between acts may not be flexible enough to test a wide range of diverse \glspl{adas}.\\
\geoscenario{}~\cite{geoScenario} is a based on the \gls{osm}~\cite{openStreetMap} standard \gls{dsl} to formalize test scenarios.
The goal of this formalization is to introduce reproducibility of test cases by enabling testers to describe complex traffic situations.
Additionally the formalization focuses on an abstraction from the underlying tools that perform tests to allow to exchange them easily.
So \geoscenario{} allows to reuse well known existing tools like the \gls{josm}~\cite{josm} with minor changes and can easily utilize existing services like Bing Maps~\cite{bingMaps} and ESRI Maps~\cite{esriMaps}.\\
\beepbeep{}~\cite{beepbeep} is a formal and declarative \gls{dsl} for \gls{cep} queries on event traces of vehicles.
In contrast to other approaches it allows to define custom boolean queries and does not restrict \gls{cep} on a predefined set of boolean queries.
Further it allows to describe non boolean queries and compose them using temporal logic based expressions in an exclusively declarative way.
This strategy is very similar to the formalization which this work develops.
Additionally \beepbeep{} can be used for real time evaluation.

\subsection{Generation of Environments}
\scenic{}~\cite{scenic} is a probabilistic \gls{dsl} which is geared towards the creation of training and test data sets for \gls{ml} systems.
Its scenarios define an environment model that characterizes scenarios which are \eg{} realistic and describe certain types of interesting input.
In contrast to other approaches \scenic{} allows to specify distributions on the parameters that define environments instead of using concrete values.
Furthermore these environment models allow a formal analysis of the properties and the correctness of a given \gls{ml} system.\\
The \gls{sml}~\cite{sml} framework allows to automatically generate complex scenarios which have static as well as dynamic elements.
It is designed to author studies on the behavior of drivers in realistic traffic simulations.
Therefore a \gls{sml} script declares events (\eg{} accidents) that occur to a driver under predefined conditions and behavior fractions which group a set of commands (\eg{} brake, steer and accelerate).
Further it can compose these to supervisor elements which influence a simulated scenario.
Since a \gls{sml} script is a \gls{xml} file the behavior elements are fixed during a simulation which comes with the same problems as \openscenario{} at least to a certain extent.
Additionally events trigger modifications to a scenario during simulation time.
So the \gls{sml} framework requires a simulator that can change its content during the execution of a simulation.\\
An approach to implement a test generator is \gls{ac3r}~\cite{ac3r} which can reconstruct crashes by translating crash reports into simulations~\cite{crashReconstruction}.
Therefore \gls{ac3r} uses \gls{nlp} to extract information from a semi-structured police report like provided by the \gls{nhtsa} database.
The extracted information provides the geometry of the required roads as well as the initial positions of the involved participants.
To determine the trajectories that led to the crash \gls{ac3r} applies heuristics.
Finally, \gls{ac3r} generates a \beamng{} scenario that can simulate the crash.
Additionally \gls{ac3r} offers a way of measuring the accuracy of the simulation compared to the input report and is capable to derive test cases which encode similar conditions under which the initial crash happened.
This allows testers to check whether their \gls{ai} would have had an accident under the same or similar conditions too.\\
Another approach is \asfault{}~\cite{asfault} which uses procedural content generation~\cite{proceduralContentGeneration}.
\asfault{} focuses on testing lane keeping capabilities of \glspl{av}.
Therefore it generates scenarios with exactly one road which an \gls{av} has to follow to pass the test.
The curvature of the road is based on B-splines and a random number generator.\\
Another approach is evolutionary computation~\cite{generationForRacingGames} which aims to generate single lane tracks with a high diversity.
The paper measures the diversity of a track in terms of its curvature and speed profile.
The more different types of curves concerning their length, their size and their angle and the more diverse the speeds an \gls{av} can reach on different sections of the generated track the more diverse it is considered.
To achieve a high diversity the approach uses multi objective \glspl{ga} that evolve tracks with an as high as possible variety of turns and straight as well as driving speeds.

\subsection{Simulators}
\opends{}~\cite{opends} is a java based cross platform simulator which specializes to simulate \glspl{av}.
It utilizes the capabilities of the physics library \jbullet{}~\cite{jBullet} and therefore provides a very realistic behavior of participants in the simulation.
It bases the generation of the graphical representation for the simulation on the \jmonkey{}~\cite{jmonkey} framework and uses the \gls{lwjgl}~\cite{lwjgl} to utilize the \gls{gpu}.
Further \opends{} has the capability to generate scenarios from \opendrive{} formalizations.
However, \glspl{av} in \opends{} lack physical properties including a detailed damage report or the behavior of the bodywork and thus a detailed driving behavior.\\
\simulAII{}~\cite{simulA2} is a \webots{}~\cite{webots} based simulator which aims to evaluate \glspl{adas}.
To increase the realism of the simulations of \webots{} \simulAII{} introduces multiple agents that control and influence \glspl{av} regardless of any action of the user.
These agents use a communication scheme which can not only implement communication with \webots{} but also exchange of data with real cars.
Therefore \simulAII{} uses tools and libraries of the \ros{}~\cite{rosProject}.
Although the agents make the simulations more real the physical behavior of objects does not allow highly accurate test executions.\\
\carla{}~\cite{carla} is a simulator that is geared towards to simulate urban environments especially traffic.
It comes with many assets and models including buildings, vehicles, streets and weather conditions to create detailed urban scenarios.
\carla{} can only simulate vehicles and pedestrians that are abstracted by the concept of actors which represent the complete dynamic content.
It uses a server multi-client architecture that allows to distribute the control of all actors across multiple nodes.
Further it offers interfaces to manage all simulation related aspects \eg{} traffic generation, behavior of pedestrians and vehicles, weather conditions and sensors attached to any actor.
The range of available sensors includes cameras, \gls{lidar}, depth and \gls{gps} sensors.
Additionally \carla{} can create environments from \opendrive{} descriptions and integrates \gls{ros}.\\
\airsim{}~\cite{airsim} is a project of Microsoft that serves as a plugin for Unreal Engine~\cite{unrealEngine} environments.
It can simulate vehicles but its focus is the simulation of drones.
The goal of \airsim{} is to collect annotated training data for deep learning or reinforcement learning based \glspl{ai} by controlling vehicles mostly manually.
The available data includes only the state of vehicles, vision cameras and \gls{lidar} sensors.
The generation of maps for \airsim{} is a complex task since it requires to generate Unreal Engine environments.\\
\Gls{torcs}~\cite{torcs} is geared towards developing, testing and comparing \glspl{ai}.
Many papers and research oriented competitions utilize \gls{torcs} because of its high degree of modularity of the architecture and its simple vehicle model which covers only basic properties of vehicle components and mechanical details, a simple aerodynamic model and friction.
\Gls{torcs} provides a set of multiple predefined racing tracks and does not support generating new tracks.
The implementation of \glspl{ai} for \gls{torcs} is complicated and requires \glspl{ai} to control cars comprehensively \eg{} to shift gears.\\
Siemens PLM Software developed a X-in-the-loop driving simulation platform~\cite{xInTheLoop} which focuses on testing and validating \glspl{adas}.
The X under test may be software, hardware or the model of an \gls{av} or even a human driver.
The platform is based on \prescan{}, \matlab{} and \simulink{} which enables it to handle real-time models of \glspl{av} and their subsystems.
Further the simulation platform provides tactile, motion, acoustic and visual feedback which enables a human to steer a car manually in real-time.\\
\beamng{}~\cite{beamNG} is a simulator that uses in contrast to most other simulators a bottom up approach to model the physics of vehicles.
Therefore instead of defining the physics of vehicles directly it specifies physics for all separate components of a vehicle \eg{} tires, suspensions, differentials, engines, doors and glass.
A model of a vehicle is a collection of these components which are linked together with beams.
So the physics of a vehicle is the result of the physics of all its components and their connections.
Hence the physics of a vehicle has a very high level of detail and its behavior is very realistic.
\beamng{} includes numerous models for vehicles, textures for road materials with different characteristics that influence the friction and weather conditions.
There is a free but limited research version of \beamng{} which offers the full capacity of the physics engine but provides only one vehicle model and only one predefined environment.
\beamngpy{} is a Python interface to dynamically create environments, to control \beamng{} simulations and to request data about vehicles and from their sensors.
This work uses \beamng{} for all simulations.

\subsection{Simulation Infrastructure and Platforms}
To make a real car autonomous and to setup an environment to develop and test it is very time consuming, error prone and expensive.\\
\autonomoose{}~\cite{autonomoose} is a research platform at the university of Waterloo which modifies a single car to increase its level of autonomy step by step.
This car is equipped with radar, sonar, \gls{lidar}, inertial and vision sensors as well as a powerful embedded computer.
This platform subsumes multiple research groups of multiple faculties.
The current projects aim to improve the behavior in all-weather conditions, to optimize the fuel consumption, to reduce emissions and to provide methods to design safe and robust computer-based controls.\\
\deepracer{}~\cite{deepracer} is a project on the \gls{aws} platform and focuses on \gls{rl}.
Therefore it provides a software environment which allows to design, train, test and simulate \gls{rl} models.
The project also offers a small \gls{av} which has a camera and is ready to accept and test trained \gls{rl} models in the real world.
Additionally researchers can compete with each other researchers in a racing league which \gls{aws} hosts.\\
Setting up simulations and interfaces for interacting with them is a complex task and not standardized.
Existing simulators and architectural designs for simulation based testing are typically not freely available.
Nevertheless, there are platforms like \metamoto{}~\cite{metamoto} that provide a simulation infrastructure to research groups in terms of a \gls{saas} to circumvent these problems.

\subsection{Comprehensive Approaches}
\paracosm{}~\cite{paracosm} is a project which specifies a \gls{dsl} to formalize test cases plus a simulation architecture.
The \gls{dsl} is a synchronous reactive programming language whose main concept are reactive objects which bundle geometric and graphical features with the behavior over time of physical objects in a simulation.
\paracosm{} uses 3D meshes to represent the geometric features of physical objects.
Each reactive object defines input and output streams of data through which objects can communicate to each other and be composed to more complex objects in a flexible way.
Hence the actual computations on or the analysis of data are equivalent to stream transformations over objects.
\paracosm{} provides only a small set of sensor data which includes camera and depth images.
To extend \paracosm{} with further types of sensor data it has to be extended using the underlying \gls{api}.
\paracosm{} can also generate random test cases automatically.
However, \paracosm{} does not provide any constructs for specifying test criteria.
Additionally the internal representation is not compatible with any other well known simulator than the one \paracosm{} comes with.
This shipped simulator lacks a precise reflection of physical behaviors in comparison to other simulators.
The paper which presents \paracosm{} is not clear about how \glspl{ai} communicate with a simulator and there is no information about any performance measures.
The paper is also not clear about whether simulator instances can run in parallel or whether the processes of the architecture can be distributed among multiple computers.\\
The open source project \apollo{}~\cite{apollo} is a comprehensive platform which specifies a high performance and flexible architecture for the complete life cycle of developing, testing and deploying \glspl{av}.
It ships with software components to localize traffic participants, to percept the environment and to plan routes plus it supports many types of sensors \eg{} \gls{lidar} sensors, cameras, radars and ultrasonic sensors.
Additionally \apollo{} comes with a cloud service and offers a web application which visualizes the current output of relevant modules, shows the status of hardware components, offers debugging tools, activates or disables modules during a simulation and allows to control \glspl{av} manually.
However \apollo{} does not provide test case criteria which consider complete scenarios or multiple traffic participants at once \eg{} distance between participants.\\
\autoware{}~\cite{autoware, autowareOpen, autowareOnBoard} is another comprehensive open source project.
Similar to \apollo{} it builds a complete ecosystem which implements algorithms for localization, perception, detection, prediction and planning, ships with predefined maps and has the capability to handle actual hardware including sensors and vehicles.
It comes with the \gls{lgsvl} simulator that can visualize information like perception data or status of other participants.
\autoware{} works with \gls{rosbag} files~\cite{rosbag} which allow to record, replay and debug executed simulations.
The integration of \gls{rosbag} requires \autoware{} to describe environments with pixel clouds.
Hence in order to work with \autoware{} a tester has to invest lots of time to create comprehensive pixel clouds.
Furthermore a test has to rely on the shipped perception algorithm since this is the only source of information about the environment for every component.

\subsection{Implementation of \texorpdfstring{\glspl{ai}}{AI}}
To determine requirements of \glspl{ai} that control \glspl{av} in simulations this work considers multiple common approaches to implement \glspl{ai}.
The two major paradigms to implement \glspl{ai} are mediated perception and behavior reflex.
Mediated perception relies on camera images and tries to identify and classify objects as lanes, other participants, obstacles, \etc{}
Based on the extracted information an \gls{ai} computes control commands for an \gls{av}.
Behavior reflex relies on camera images as well but instead of extracting information from an image it directly maps the whole image to a driving action using a regressor like a \gls{cnn}.\\
\Gls{dave}~\cite{dave} is a behavior reflex approach and relies on images of a single front camera.
It uses a \gls{cnn} to compute steering commands directly from given input images.
This \gls{cnn} uses camera images which are mapped to human steering angles as its training data.
It learns to detect useful road features on its own and does not include any preprocessing steps for the camera images.
The paper claims that this approach eventually leads to better performance and smaller systems because the internal components self-optimize.\\
Another approach for implementing an \gls{ai} is \deepdriving{}~\cite{deepDriving}.
\deepdriving{} is a direct perception approach which lies in between mediated perception and behavior reflex approaches.
Unlike mediated perception it does not extract as much information as possible but uses only a small number of key perception indicators from an input image and estimates an affordance value of the situation at which the image was taken.
The indicators include distances to lane markings and preceding participants.
To extract the key perception indicators the approach uses deep \glspl{cnn} which are trained by manually driving an \gls{av} for a few hours.
The paper claims that the set of key perception indicators is sufficient to completely describe scenes and to base driving decisions on it.\\
Another approach is deep \gls{rl}~\cite{learnToDriveInADay}.
This approach aims to train a neural network model that focuses solely on the lane keeping problem.
Therefore it translates the problem of training a neural network into a problem of solving \glspl{mdp}.
The input of the neural network are single monocular images and the distance the \gls{av} traveled safely defines the reward for the neural network.
The approach determines the traveled distance by feature extraction of the input image.
The paper claims that it uses a continues and model-free algorithm which can perform all exploration and optimizations on the \gls{av} while it drives.
Further it claims that an \gls{av} learns to follow a lane with only a few steps of training.\\
\scaleai{}~\cite{scaleAI} is a company which offers training data for \glspl{av} to companies including Samsung, Toyota and lyft.
It offers data as images and videos.
The types of available data are \gls{lidar} sensors, radar, point clouds, traffic lights plus annotated and labeled data.
The data can be requested via a \gls{rest} \gls{api}.
This \gls{api} also offers \gls{ml} based methods to label and annotate custom images and videos.
