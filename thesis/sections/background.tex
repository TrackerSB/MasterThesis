%!TEX root = ../thesis.tex
\section{Background}\label{sec:background}
A common execution strategy for simulations is \textbf{synchronous simulation}.
This strategy avoids simulations to be influenced by network latency and the current load of the underlying hardware.
As a drawback this strategy does not support real-time simulations.\\
This strategy makes a simulation process wait frequently for \gls{ai} processes which control \glspl{av} in the simulation to reach a certain state.
Only if the \gls{ai} processes reach this state both the simulation process and the \gls{ai} processes continue.\par

Since synchronous simulation makes simulations pause frequently the real time does not correspond to their simulated time.
Further is the speed of the simulated time dependent on the current load of the underlying hardware.
To solve this the simulation time is separated from the real time by using \textbf{Ticks}~\cite{tickrate}.
Ticks define logical time units~\cite{logicalTime} on which both simulations and test cases in my work are based on.
A single tick is a time interval of predefined length in which a simulator calculates the changes to the environment plus to all traffic participants and applies them to the simulation.
A tick is the smallest considered logical time unit and can not be divided.\par

The \textbf{three-way handshake}~\cite{computernetzwerke} is a communication protocol that allows to establish a connection between a client and a server on an unreliable channel.
Therefore the initiating client sends a \code{SYN} (\enquote{synchronize}) package to the server to request a connection.
The server responds with a \code{SYN-ACK} (\enquote{\code{SYN} acknowledge}) package that informs the client that the server opened a connection.
The client answers this package again with a \code{ACK} (\enquote{acknowledge}) package which confirms that the client knows about the opened connection.
This establishes a connection.\\
The most well-known application is \gls{tcp} which extends the protocol with additional properties including error detection.
\Gls{tcp} builds the basis of \gls{http} requests.
Nevertheless, a three-way handshake is suitable to implement simple request mechanisms to exchange data on unreliable channels.\par

\textbf{Micro services} is an architectural design pattern which splits an application into small and autonomous services and connects them with light weight protocols~\cite{microservices}.
In contrast to other architectural design patterns like \gls{soa} micro services introduce a very high level of resilience, the possibility to scale services independently instead of the whole application and easier composition of heterogeneous technologies.
As a drawback micro services increase the complexity of the development and the deployment of an application.
Further it requires more communication between its components which the application more dependent on network latencies.\par

Complex objects can often not be stored or transfered over a network as they are.
\textbf{Serialization} is a technique to convert a complex object to a textual or binary representation and back~\cite{serialization}.
These textual or binary representations can be stored or transfered over a network.
Further serialization may implement additional properties which allow to check whether a serialized object is broken \eg{} after transferring it over an unreliable connection.\par

The \textbf{master slave pattern} is a communication model which allows to distribute similar or identical computations over multiple nodes and parallelize them~\cite{masterSlave}.
The pattern consists of a number of slave nodes that do the actual computations and exactly one master node that distributes tasks among the slave nodes, organizes them and collects their results.
On the one hand this architecture is simple and introduces tolerance for faults of computations on slave nodes.
On the other hand it also introduces the master node as single point of failure.
Further this architecture does not allow direct communication between slave nodes without involving the master node which increases the required communication in case the slave nodes need to exchange data.
